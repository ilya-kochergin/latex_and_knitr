% будем использвать xelatex как программу 
% компиляции из latex в pdf

\documentclass{article}
\usepackage{xltxtra}
\usepackage{fontspec}
\usepackage{polyglossia}
\usepackage{minted}   % листинги кода с подсветкой синтаксиса при помощи внешней программы, основанной на Pygments 
% => при компиляции нужно добавить опцию --shell-escape к командной строке xelatex (pdflatex или lualatex)
%  -- выдаёт ошибку, если включить пакет после следующего оператора 
\setmainlanguage[babelshorthands=true]{russian} 
% это установки к пакету polyglossia
% 
% babel shorthands are:
%  "--- Cyrillic emdash in plain text.
%  "--~ Cyrillic emdash in compound names (surnames).
%  "--* Cyrillic emdash for denoting direct speech.
% "~ неразрывный дефис. Пример: как в 90"~х годах
\setotherlanguage{english}


\defaultfontfeatures{Renderer=Basic, Ligatures=TeX} %лигатуры работают в basic режиме, например, <<кавычки-ёлочки>> или ,,кавычки-лапки``. 
% \defaultfontfeatures{Scale=MatchLowercase, Mapping=tex-text} 

\setmainfont{Times New Roman}
\setsansfont{Arial}
\setmonofont{Courier New}




\newfontfamily{\cyrillicfont}{Times New Roman}
\newfontfamily{\cyrillicfonttt}{Courier New} 
%\newfontfamily{\cyrillicfonttt}{Monaco} % на Маке


% \usepackage{tikz}
% \usetikzlibrary{graphs,shapes,arrows}
 
\author{Илья Кочергин}
\title{Решение конфликта параметра \textsf{babelshorthands=true} при включении русского языка и пакета для печати программных листингов \textsf{minted} в документе \XeLaTeX{}.    }


\begin{document}
	\maketitle
	
	\begin{abstract}
	     Документ демонстрирует возможность решения конфликта пакета minted и параметра babelshorthands. Ошибка компиляции, которую вызывает сочетание этих установок, может быть устранена путем переноса команды, включающий данных пакет,
	     \mint{tex}|\usepackage{minted}|
	      выше по тексту, чем команда \mint{tex}|\setmainlanguage[babelshorthands=true]{russian}|
	\end{abstract}
	
\section{Почему \XeLaTeX а не {\small pdf}\LaTeX{}? }
	
Документ скомпилирован при помощи \XeLaTeX. Для компиляции из языка разметки \LaTeX{}  в формат \emph{PDF}  можно было бы использовать классический  {\scriptsize pdf}\LaTeX{} или, наоборот, самый новаторский \LuaLaTeX. Мы выбираем \XeLaTeX{} за 3~достижения:

\begin{enumerate}
	\item  Полная и встроенная  в систему поддержка употребления многих языков в одном документе и использование кодировки Unicode (UTF"~8).
	\item  Поддержка шрифтов операционной системы. Можете использовать, например, привычные  \textbf{\rmfamily Times New Roman}, \textbf{\sffamily Arial} и \textbf{\texttt{ Courier New}} как, соответственно, шрифтов с засечками (\textbf{r}o\textbf{m}an "---\textbf{rm}, serif), {\sffamily рубленный (\textbf{s}an seri\textbf{f} "--- sf)} и {\ttfamily моноширинный (monospace, \textbf{t}ele\textbf{t}ype "--- \textbf{tt})}  шрифты. 
	Напомним стандартные кыоманды для их включения: 
		\mint{tex}|\textrm{тест}, \textsf{тест}, \texttt{тест}|
		
		
	\item  Быстрая, по сравнению с \LuaLaTeX, компиляция. 
	
\end{enumerate}


\section{Зачем пакет minted?}

Наиболее продвинутая система подсветки синтаксиса  во фрагментах программного кода использует python-библиотеку Pygments, доступную в  \LaTeX{} посредством пакета \texttt{minted}:

\usemintedstyle{tango}


\mint[bgcolor=gray!5]{tex}|$\dfrac{\sin x}{x} $ % --это комментарий  до конца строки|

\bigskip{}
\medskip

А вот более длинный листинг программного кода 

\begin{minted}[linenos,bgcolor=gray!5!yellow!2,
   frame=lines, stepnumber=2,tabsize=4,obeytabs=false]{R}
z <- 5
while(z >= 3 && z <= 10) {
   print(z)
   coin <- rbinom(1, 1, 0.5)
   if(coin == 1) { # случайное блуждание
       z <- z + 1
   } else {
       z <- z - 1
   }
}
\end{minted} 
\\
на языке R.

\section{Зачем сочетания символов в стиле пакета babel?}

Теперь применяем сочетания символов в стиле \textsf{babel}: 

"--* Кавычка и тройное тире  "--- это  <<бабель-идиома>>  для печати тире в стиле русско-немецких типографских традиций.

"--* Ну, ты сказал!  

"--* Это не я сказал. Это "--- Лебедев"--~Щукин.



\end{document}
